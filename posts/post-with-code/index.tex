% Options for packages loaded elsewhere
\PassOptionsToPackage{unicode}{hyperref}
\PassOptionsToPackage{hyphens}{url}
\PassOptionsToPackage{dvipsnames,svgnames,x11names}{xcolor}
%
\documentclass[
  letterpaper,
  DIV=11,
  numbers=noendperiod]{scrreprt}

\usepackage{amsmath,amssymb}
\usepackage{iftex}
\ifPDFTeX
  \usepackage[T1]{fontenc}
  \usepackage[utf8]{inputenc}
  \usepackage{textcomp} % provide euro and other symbols
\else % if luatex or xetex
  \usepackage{unicode-math}
  \defaultfontfeatures{Scale=MatchLowercase}
  \defaultfontfeatures[\rmfamily]{Ligatures=TeX,Scale=1}
\fi
\usepackage{lmodern}
\ifPDFTeX\else  
    % xetex/luatex font selection
\fi
% Use upquote if available, for straight quotes in verbatim environments
\IfFileExists{upquote.sty}{\usepackage{upquote}}{}
\IfFileExists{microtype.sty}{% use microtype if available
  \usepackage[]{microtype}
  \UseMicrotypeSet[protrusion]{basicmath} % disable protrusion for tt fonts
}{}
\makeatletter
\@ifundefined{KOMAClassName}{% if non-KOMA class
  \IfFileExists{parskip.sty}{%
    \usepackage{parskip}
  }{% else
    \setlength{\parindent}{0pt}
    \setlength{\parskip}{6pt plus 2pt minus 1pt}}
}{% if KOMA class
  \KOMAoptions{parskip=half}}
\makeatother
\usepackage{xcolor}
\setlength{\emergencystretch}{3em} % prevent overfull lines
\setcounter{secnumdepth}{-\maxdimen} % remove section numbering
% Make \paragraph and \subparagraph free-standing
\makeatletter
\ifx\paragraph\undefined\else
  \let\oldparagraph\paragraph
  \renewcommand{\paragraph}{
    \@ifstar
      \xxxParagraphStar
      \xxxParagraphNoStar
  }
  \newcommand{\xxxParagraphStar}[1]{\oldparagraph*{#1}\mbox{}}
  \newcommand{\xxxParagraphNoStar}[1]{\oldparagraph{#1}\mbox{}}
\fi
\ifx\subparagraph\undefined\else
  \let\oldsubparagraph\subparagraph
  \renewcommand{\subparagraph}{
    \@ifstar
      \xxxSubParagraphStar
      \xxxSubParagraphNoStar
  }
  \newcommand{\xxxSubParagraphStar}[1]{\oldsubparagraph*{#1}\mbox{}}
  \newcommand{\xxxSubParagraphNoStar}[1]{\oldsubparagraph{#1}\mbox{}}
\fi
\makeatother


\providecommand{\tightlist}{%
  \setlength{\itemsep}{0pt}\setlength{\parskip}{0pt}}\usepackage{longtable,booktabs,array}
\usepackage{calc} % for calculating minipage widths
% Correct order of tables after \paragraph or \subparagraph
\usepackage{etoolbox}
\makeatletter
\patchcmd\longtable{\par}{\if@noskipsec\mbox{}\fi\par}{}{}
\makeatother
% Allow footnotes in longtable head/foot
\IfFileExists{footnotehyper.sty}{\usepackage{footnotehyper}}{\usepackage{footnote}}
\makesavenoteenv{longtable}
\usepackage{graphicx}
\makeatletter
\newsavebox\pandoc@box
\newcommand*\pandocbounded[1]{% scales image to fit in text height/width
  \sbox\pandoc@box{#1}%
  \Gscale@div\@tempa{\textheight}{\dimexpr\ht\pandoc@box+\dp\pandoc@box\relax}%
  \Gscale@div\@tempb{\linewidth}{\wd\pandoc@box}%
  \ifdim\@tempb\p@<\@tempa\p@\let\@tempa\@tempb\fi% select the smaller of both
  \ifdim\@tempa\p@<\p@\scalebox{\@tempa}{\usebox\pandoc@box}%
  \else\usebox{\pandoc@box}%
  \fi%
}
% Set default figure placement to htbp
\def\fps@figure{htbp}
\makeatother

\KOMAoption{captions}{tableheading}
\makeatletter
\@ifpackageloaded{tcolorbox}{}{\usepackage[skins,breakable]{tcolorbox}}
\@ifpackageloaded{fontawesome5}{}{\usepackage{fontawesome5}}
\definecolor{quarto-callout-color}{HTML}{909090}
\definecolor{quarto-callout-note-color}{HTML}{0758E5}
\definecolor{quarto-callout-important-color}{HTML}{CC1914}
\definecolor{quarto-callout-warning-color}{HTML}{EB9113}
\definecolor{quarto-callout-tip-color}{HTML}{00A047}
\definecolor{quarto-callout-caution-color}{HTML}{FC5300}
\definecolor{quarto-callout-color-frame}{HTML}{acacac}
\definecolor{quarto-callout-note-color-frame}{HTML}{4582ec}
\definecolor{quarto-callout-important-color-frame}{HTML}{d9534f}
\definecolor{quarto-callout-warning-color-frame}{HTML}{f0ad4e}
\definecolor{quarto-callout-tip-color-frame}{HTML}{02b875}
\definecolor{quarto-callout-caution-color-frame}{HTML}{fd7e14}
\makeatother
\makeatletter
\@ifpackageloaded{caption}{}{\usepackage{caption}}
\AtBeginDocument{%
\ifdefined\contentsname
  \renewcommand*\contentsname{Table of contents}
\else
  \newcommand\contentsname{Table of contents}
\fi
\ifdefined\listfigurename
  \renewcommand*\listfigurename{List of Figures}
\else
  \newcommand\listfigurename{List of Figures}
\fi
\ifdefined\listtablename
  \renewcommand*\listtablename{List of Tables}
\else
  \newcommand\listtablename{List of Tables}
\fi
\ifdefined\figurename
  \renewcommand*\figurename{Figure}
\else
  \newcommand\figurename{Figure}
\fi
\ifdefined\tablename
  \renewcommand*\tablename{Table}
\else
  \newcommand\tablename{Table}
\fi
}
\@ifpackageloaded{float}{}{\usepackage{float}}
\floatstyle{ruled}
\@ifundefined{c@chapter}{\newfloat{codelisting}{h}{lop}}{\newfloat{codelisting}{h}{lop}[chapter]}
\floatname{codelisting}{Listing}
\newcommand*\listoflistings{\listof{codelisting}{List of Listings}}
\usepackage{amsthm}
\theoremstyle{definition}
\newtheorem{exercise}{Exercise}[section]
\theoremstyle{plain}
\newtheorem{proposition}{Proposition}[section]
\theoremstyle{plain}
\newtheorem{theorem}{Theorem}[section]
\theoremstyle{definition}
\newtheorem{definition}{Definition}[section]
\theoremstyle{plain}
\newtheorem{conjecture}{Conjecture}[section]
\theoremstyle{plain}
\newtheorem{corollary}{Corollary}[section]
\theoremstyle{remark}
\AtBeginDocument{\renewcommand*{\proofname}{Proof}}
\newtheorem*{remark}{Remark}
\newtheorem*{solution}{Solution}
\newtheorem{refremark}{Remark}[section]
\newtheorem{refsolution}{Solution}[section]
\makeatother
\makeatletter
\makeatother
\makeatletter
\@ifpackageloaded{caption}{}{\usepackage{caption}}
\@ifpackageloaded{subcaption}{}{\usepackage{subcaption}}
\makeatother

\usepackage{bookmark}

\IfFileExists{xurl.sty}{\usepackage{xurl}}{} % add URL line breaks if available
\urlstyle{same} % disable monospaced font for URLs
\hypersetup{
  pdftitle={Post With Code},
  pdfauthor={Harlow Malloc},
  colorlinks=true,
  linkcolor={blue},
  filecolor={Maroon},
  citecolor={Blue},
  urlcolor={Blue},
  pdfcreator={LaTeX via pandoc}}


\title{Post With Code}
\author{Harlow Malloc}
\date{2025-11-08}

\begin{document}
\maketitle


This is a post with executable code.

\begin{tcolorbox}[enhanced jigsaw, opacityback=0, opacitybacktitle=0.6, bottomtitle=1mm, colbacktitle=quarto-callout-note-color!10!white, coltitle=black, colback=white, left=2mm, colframe=quarto-callout-note-color-frame, bottomrule=.15mm, arc=.35mm, titlerule=0mm, breakable, rightrule=.15mm, toprule=.15mm, toptitle=1mm, title=\textcolor{quarto-callout-note-color}{\faInfo}\hspace{0.5em}{Definition (Important definition)}, leftrule=.75mm]

\begin{definition}[Important
definition]\protect\hypertarget{def-first}{}\label{def-first}

A number \(a\) is called \emph{positivie} if \(a>0\).

\end{definition}

\end{tcolorbox}

\begin{tcolorbox}[enhanced jigsaw, bottomrule=.15mm, opacityback=0, breakable, toprule=.15mm, rightrule=.15mm, arc=.35mm, colback=white, left=2mm, leftrule=.75mm, colframe=quarto-callout-note-color-frame]

\begin{theorem}[Important
theorem]\protect\hypertarget{thm-first}{}\label{thm-first}

If \(a>b\) and \(b>c\) then \(a>c\).

\end{theorem}

\end{tcolorbox}

\begin{tcolorbox}[enhanced jigsaw, opacityback=0, opacitybacktitle=0.6, bottomtitle=1mm, colbacktitle=quarto-callout-tip-color!10!white, coltitle=black, colback=white, left=2mm, colframe=quarto-callout-tip-color-frame, bottomrule=.15mm, arc=.35mm, titlerule=0mm, breakable, rightrule=.15mm, toprule=.15mm, toptitle=1mm, title=\textcolor{quarto-callout-tip-color}{\faLightbulb}\hspace{0.5em}{Remark (Important remark)}, leftrule=.75mm]

\begin{refremark}[Important remark]
The property in Theorem~\ref{thm-first} is called ``transitivity'\,'.

\label{rem-first}

\end{refremark}

\end{tcolorbox}

\phantomsection\label{nt-first}
\section{Commutation relations}\label{commutation-relations}

We are going to discuss now commutation relations.

This example Quarto markdown file demonstrates the use of the
\texttt{callouty-theorem} filter.

\subsection{Examples}\label{examples}

\begin{tcolorbox}[enhanced jigsaw, bottomrule=.15mm, opacityback=0, breakable, toprule=.15mm, rightrule=.15mm, arc=.35mm, colback=white, left=2mm, leftrule=.75mm, colframe=quarto-callout-note-color-frame]

\begin{proposition}[]\protect\hypertarget{prp-pr-number}{}\label{prp-pr-number}

If there exists a primitive root modulo \(n\), then there are exactly
\(\varphi(\varphi(n))\) primitive roots modulo \(n\).

\end{proposition}

\end{tcolorbox}

\begin{tcolorbox}[enhanced jigsaw, bottomrule=.15mm, opacityback=0, breakable, toprule=.15mm, rightrule=.15mm, arc=.35mm, colback=white, left=2mm, leftrule=.75mm, colframe=quarto-callout-note-color-frame]

\begin{theorem}[Existence of primitive
roots]\protect\hypertarget{thm-pr-existence}{}\label{thm-pr-existence}

Primitive roots modulo \(n\) exists if and only if
\(n = 2, 4, p^k, 2p^k\) for an odd prime \(p\) and a positive integer
\(k\).

\end{theorem}

\end{tcolorbox}

\begin{tcolorbox}[enhanced jigsaw, opacityback=0, opacitybacktitle=0.6, bottomtitle=1mm, colbacktitle=quarto-callout-note-color!10!white, coltitle=black, colback=white, left=2mm, colframe=quarto-callout-note-color-frame, bottomrule=.15mm, arc=.35mm, titlerule=0mm, breakable, rightrule=.15mm, toprule=.15mm, toptitle=1mm, title=\textcolor{quarto-callout-note-color}{\faInfo}\hspace{0.5em}{Proof (Proof of Proposition~\ref{prp-pr-number})}, leftrule=.75mm]

\begin{proof}[Proof of Proposition~\ref{prp-pr-number}]
We note that the primitive roots modulo \(n\) is exactly the generators
of the group of units modulo \(n\). By the hypothesis, the group of
units modulo \(n\) is cyclic, thus having \(\varphi(\varphi(n))\)
generators.
\end{proof}

\end{tcolorbox}

\begin{tcolorbox}[enhanced jigsaw, opacityback=0, opacitybacktitle=0.6, bottomtitle=1mm, colbacktitle=quarto-callout-tip-color!10!white, coltitle=black, colback=white, left=2mm, colframe=quarto-callout-tip-color-frame, bottomrule=.15mm, arc=.35mm, titlerule=0mm, breakable, rightrule=.15mm, toprule=.15mm, toptitle=1mm, title=\textcolor{quarto-callout-tip-color}{\faLightbulb}\hspace{0.5em}{Remark}, leftrule=.75mm]

\begin{remark}
Group theory greatly simplifies the proof of the theorem.
\end{remark}

\end{tcolorbox}

\begin{tcolorbox}[enhanced jigsaw, bottomrule=.15mm, opacityback=0, breakable, toprule=.15mm, rightrule=.15mm, arc=.35mm, colback=white, left=2mm, leftrule=.75mm, colframe=quarto-callout-tip-color-frame]

\begin{exercise}[]\protect\hypertarget{exr-quadratic-residue}{}\label{exr-quadratic-residue}

Prove that the quadratic residues modulo \(p\) form a subgroup of the
group of units modulo \(p\) of index \(2\).

\end{exercise}

\end{tcolorbox}

\begin{tcolorbox}[enhanced jigsaw, opacityback=0, opacitybacktitle=0.6, bottomtitle=1mm, colbacktitle=quarto-callout-note-color!10!white, coltitle=black, colback=white, left=2mm, colframe=quarto-callout-note-color-frame, bottomrule=.15mm, arc=.35mm, titlerule=0mm, breakable, rightrule=.15mm, toprule=.15mm, toptitle=1mm, title=\textcolor{quarto-callout-note-color}{\faInfo}\hspace{0.5em}{Solution (Solution to Exercise~\ref{exr-quadratic-residue})}, leftrule=.75mm]

\begin{refsolution}[Solution to Exercise~\ref{exr-quadratic-residue}]
Use the fact that the group of units modulo \(p\) is cyclic.

\label{sol-quadratic-residue}

\end{refsolution}

\end{tcolorbox}

\subsection{On default behaviors}\label{on-default-behaviors}

\begin{tcolorbox}[enhanced jigsaw, opacityback=0, opacitybacktitle=0.6, bottomtitle=1mm, colbacktitle=quarto-callout-note-color!10!white, coltitle=black, colback=white, left=2mm, colframe=quarto-callout-note-color-frame, bottomrule=.15mm, arc=.35mm, titlerule=0mm, breakable, rightrule=.15mm, toprule=.15mm, toptitle=1mm, title=\textcolor{quarto-callout-note-color}{\faInfo}\hspace{0.5em}{Note}, leftrule=.75mm]

\begin{corollary}[Default
style]\protect\hypertarget{cor-default}{}\label{cor-default}

If you set the metadata of a theorem type to \texttt{default}, it will
be rendered like this.

\end{corollary}

\end{tcolorbox}

\begin{tcolorbox}[enhanced jigsaw, opacityback=0, opacitybacktitle=0.6, bottomtitle=1mm, colbacktitle=quarto-callout-note-color!10!white, coltitle=black, colback=white, left=2mm, colframe=quarto-callout-note-color-frame, bottomrule=.15mm, arc=.35mm, titlerule=0mm, breakable, rightrule=.15mm, toprule=.15mm, toptitle=1mm, title=\textcolor{quarto-callout-note-color}{\faInfo}\hspace{0.5em}{Definition (Default style without title)}, leftrule=.75mm]

\begin{definition}[Default style without
title]\protect\hypertarget{def-default}{}\label{def-default}

\texttt{callout} can also be set to \texttt{default} in the metadata.

\end{definition}

\end{tcolorbox}

\begin{conjecture}[As is]\protect\hypertarget{cnj-nil}{}\label{cnj-nil}

Theorem types not specified in the metadata will be rendered as is.

\end{conjecture}




\end{document}
