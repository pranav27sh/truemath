% Options for packages loaded elsewhere
\PassOptionsToPackage{unicode}{hyperref}
\PassOptionsToPackage{hyphens}{url}
\PassOptionsToPackage{dvipsnames,svgnames,x11names}{xcolor}
%
\documentclass[
  letterpaper,
  DIV=11,
  numbers=noendperiod]{scrartcl}

\usepackage{amsmath,amssymb}
\usepackage{iftex}
\ifPDFTeX
  \usepackage[T1]{fontenc}
  \usepackage[utf8]{inputenc}
  \usepackage{textcomp} % provide euro and other symbols
\else % if luatex or xetex
  \usepackage{unicode-math}
  \defaultfontfeatures{Scale=MatchLowercase}
  \defaultfontfeatures[\rmfamily]{Ligatures=TeX,Scale=1}
\fi
\usepackage{lmodern}
\ifPDFTeX\else  
    % xetex/luatex font selection
\fi
% Use upquote if available, for straight quotes in verbatim environments
\IfFileExists{upquote.sty}{\usepackage{upquote}}{}
\IfFileExists{microtype.sty}{% use microtype if available
  \usepackage[]{microtype}
  \UseMicrotypeSet[protrusion]{basicmath} % disable protrusion for tt fonts
}{}
\makeatletter
\@ifundefined{KOMAClassName}{% if non-KOMA class
  \IfFileExists{parskip.sty}{%
    \usepackage{parskip}
  }{% else
    \setlength{\parindent}{0pt}
    \setlength{\parskip}{6pt plus 2pt minus 1pt}}
}{% if KOMA class
  \KOMAoptions{parskip=half}}
\makeatother
\usepackage{xcolor}
\setlength{\emergencystretch}{3em} % prevent overfull lines
\setcounter{secnumdepth}{1}
% Make \paragraph and \subparagraph free-standing
\makeatletter
\ifx\paragraph\undefined\else
  \let\oldparagraph\paragraph
  \renewcommand{\paragraph}{
    \@ifstar
      \xxxParagraphStar
      \xxxParagraphNoStar
  }
  \newcommand{\xxxParagraphStar}[1]{\oldparagraph*{#1}\mbox{}}
  \newcommand{\xxxParagraphNoStar}[1]{\oldparagraph{#1}\mbox{}}
\fi
\ifx\subparagraph\undefined\else
  \let\oldsubparagraph\subparagraph
  \renewcommand{\subparagraph}{
    \@ifstar
      \xxxSubParagraphStar
      \xxxSubParagraphNoStar
  }
  \newcommand{\xxxSubParagraphStar}[1]{\oldsubparagraph*{#1}\mbox{}}
  \newcommand{\xxxSubParagraphNoStar}[1]{\oldsubparagraph{#1}\mbox{}}
\fi
\makeatother


\providecommand{\tightlist}{%
  \setlength{\itemsep}{0pt}\setlength{\parskip}{0pt}}\usepackage{longtable,booktabs,array}
\usepackage{calc} % for calculating minipage widths
% Correct order of tables after \paragraph or \subparagraph
\usepackage{etoolbox}
\makeatletter
\patchcmd\longtable{\par}{\if@noskipsec\mbox{}\fi\par}{}{}
\makeatother
% Allow footnotes in longtable head/foot
\IfFileExists{footnotehyper.sty}{\usepackage{footnotehyper}}{\usepackage{footnote}}
\makesavenoteenv{longtable}
\usepackage{graphicx}
\makeatletter
\newsavebox\pandoc@box
\newcommand*\pandocbounded[1]{% scales image to fit in text height/width
  \sbox\pandoc@box{#1}%
  \Gscale@div\@tempa{\textheight}{\dimexpr\ht\pandoc@box+\dp\pandoc@box\relax}%
  \Gscale@div\@tempb{\linewidth}{\wd\pandoc@box}%
  \ifdim\@tempb\p@<\@tempa\p@\let\@tempa\@tempb\fi% select the smaller of both
  \ifdim\@tempa\p@<\p@\scalebox{\@tempa}{\usebox\pandoc@box}%
  \else\usebox{\pandoc@box}%
  \fi%
}
% Set default figure placement to htbp
\def\fps@figure{htbp}
\makeatother

\KOMAoption{captions}{tableheading}
\makeatletter
\@ifpackageloaded{tcolorbox}{}{\usepackage[skins,breakable]{tcolorbox}}
\@ifpackageloaded{fontawesome5}{}{\usepackage{fontawesome5}}
\definecolor{quarto-callout-color}{HTML}{909090}
\definecolor{quarto-callout-note-color}{HTML}{0758E5}
\definecolor{quarto-callout-important-color}{HTML}{CC1914}
\definecolor{quarto-callout-warning-color}{HTML}{EB9113}
\definecolor{quarto-callout-tip-color}{HTML}{00A047}
\definecolor{quarto-callout-caution-color}{HTML}{FC5300}
\definecolor{quarto-callout-color-frame}{HTML}{acacac}
\definecolor{quarto-callout-note-color-frame}{HTML}{4582ec}
\definecolor{quarto-callout-important-color-frame}{HTML}{d9534f}
\definecolor{quarto-callout-warning-color-frame}{HTML}{f0ad4e}
\definecolor{quarto-callout-tip-color-frame}{HTML}{02b875}
\definecolor{quarto-callout-caution-color-frame}{HTML}{fd7e14}
\makeatother
\makeatletter
\@ifpackageloaded{caption}{}{\usepackage{caption}}
\AtBeginDocument{%
\ifdefined\contentsname
  \renewcommand*\contentsname{Table of contents}
\else
  \newcommand\contentsname{Table of contents}
\fi
\ifdefined\listfigurename
  \renewcommand*\listfigurename{List of Figures}
\else
  \newcommand\listfigurename{List of Figures}
\fi
\ifdefined\listtablename
  \renewcommand*\listtablename{List of Tables}
\else
  \newcommand\listtablename{List of Tables}
\fi
\ifdefined\figurename
  \renewcommand*\figurename{Figure}
\else
  \newcommand\figurename{Figure}
\fi
\ifdefined\tablename
  \renewcommand*\tablename{Table}
\else
  \newcommand\tablename{Table}
\fi
}
\@ifpackageloaded{float}{}{\usepackage{float}}
\floatstyle{ruled}
\@ifundefined{c@chapter}{\newfloat{codelisting}{h}{lop}}{\newfloat{codelisting}{h}{lop}[chapter]}
\floatname{codelisting}{Listing}
\newcommand*\listoflistings{\listof{codelisting}{List of Listings}}
\usepackage{amsthm}
\theoremstyle{plain}
\newtheorem{lemma}{Lemma}[section]
\theoremstyle{plain}
\newtheorem{corollary}{Corollary}[section]
\theoremstyle{plain}
\newtheorem{theorem}{Theorem}[section]
\theoremstyle{definition}
\newtheorem{definition}{Definition}[section]
\theoremstyle{remark}
\AtBeginDocument{\renewcommand*{\proofname}{Proof}}
\newtheorem*{remark}{Remark}
\newtheorem*{solution}{Solution}
\newtheorem{refremark}{Remark}[section]
\newtheorem{refsolution}{Solution}[section]
\makeatother
\makeatletter
\makeatother
\makeatletter
\@ifpackageloaded{caption}{}{\usepackage{caption}}
\@ifpackageloaded{subcaption}{}{\usepackage{subcaption}}
\makeatother

\usepackage{bookmark}

\IfFileExists{xurl.sty}{\usepackage{xurl}}{} % add URL line breaks if available
\urlstyle{same} % disable monospaced font for URLs
\hypersetup{
  pdftitle={Galois Theory - Prerequisites},
  pdfauthor={Pranav Shukla},
  colorlinks=true,
  linkcolor={blue},
  filecolor={Maroon},
  citecolor={Blue},
  urlcolor={Blue},
  pdfcreator={LaTeX via pandoc}}


\title{Galois Theory - Prerequisites}
\author{Pranav Shukla}
\date{2026-01-22}

\begin{document}
\maketitle


\section*{Introduction}\label{introduction}
\addcontentsline{toc}{section}{Introduction}

Hello everyone. I have recently decided to start with Galois Theory in
view of my long term goal of studying Algebraic Number Theory. Now maybe
I am being a bit desperate here since for Galois Theory one also needs a
good understanding of field theory which for me, is a bit rusty at the
moment. I have taken a course in introductory abstract algebra at my
university in my previous semester (IIIrd sem), which was fairly basic,
was taught upto ring theory, also excluding some important portions of
group and ring theory. The book that was followed was Gallian. My goal
in this post is to revise (and record) the key results, theorems and
ideas in ring and field theory especially, which will be of prime
importance for starting with Galois Theory. This will be just a
collection of theorems and results and I would not include proofs of
them here (maybe will add them gradually in the future), as it takes
much more time. Also, not having the proofs written down will allow me
to try and recall the proof strategy each time I will have to convince
myself of the theorem, improving my understanding of the proof in the
process.

\paragraph*{The Plan}\label{the-plan}
\addcontentsline{toc}{paragraph}{The Plan}

My plan is that for the revision upto field theory, I will be following
Gallian. After reaching there, I will switch to Dummit and Foote for the
Galois Theory itself. I have also enrolled in the NPTEL course for
Galois Theory this sem, which is being taught by Prof.~Krishna
Hanumanthu and I will be following that course closely too (Link of the
playlist
\href{https://youtube.com/playlist?list=PLyqSpQzTE6M94LuHxxu4OrViX4K45oH73&si=-fRFCp29_IdCycR3}{here}).
Pretty excited for what lies ahead!

\begin{itemize}
\tightlist
\item
  Edit: A minor change in the plan. Will be studying field theory mainly
  from Dummit and Foote. Target day for completion of field theory:
  \textbf{3rd of Febraury}
\end{itemize}

\section{RING THEORY}\label{ring-theory}

\subsubsection*{Factorization of
Polynomials}\label{factorization-of-polynomials}
\addcontentsline{toc}{subsubsection}{Factorization of Polynomials}

~

\begin{tcolorbox}[enhanced jigsaw, colframe=quarto-callout-note-color-frame, opacityback=0, leftrule=.75mm, arc=.35mm, colback=white, rightrule=.15mm, toprule=.15mm, bottomrule=.15mm, breakable, left=2mm]

\begin{definition}[Irreducible
element]\protect\hypertarget{def-irreducible}{}\label{def-irreducible}

~

\begin{itemize}
\item
  \textbf{In Integral Domains:} Let \(D\) be an integral domain. A
  polynomial \(f(x)\) from \(D[x]\) which is a non zero and non unit in
  \(D[x]\) is said to be \textbf{irreducible} \emph{over} \(D\) if
  \(f(x) = g(x)h(x)\), where \(g(x) \; and \; h(x)\) are in \(D[x]\)
  implies that either \(g(x) \; or \; f(x)\) is a unit. A non-zero non
  unit polynomial in \(D[x]\) is said to be reducible if it is not
  irreducible over \(D[x]\).
\item
  \textbf{In Fields:} Let \(F\) be a field. Then \(f(x)\), a non-zero
  non unit element in \(F[x]\), is irreducible over \(F\) if there exist
  \(g(x) \; and \; h(x)\) in \(F[x]\) such that \(f(x) = g(x)h(x)\)
  implies that \(deg\;h(x) < deg\;f(x) \; and\; deg\;g(x) < deg\;f(x)\).
\end{itemize}

\end{definition}

\end{tcolorbox}

\begin{tcolorbox}[enhanced jigsaw, colframe=quarto-callout-tip-color-frame, opacityback=0, leftrule=.75mm, arc=.35mm, colback=white, rightrule=.15mm, toprule=.15mm, bottomrule=.15mm, breakable, left=2mm]

\begin{remark}
\leavevmode

\begin{itemize}
\tightlist
\item
  Irreducibility is a property of the polynomial as well as the integral
  domain or the field being considered.
\item
  One can check that the defintion of irreducibility in a field is
  equivalent to that in an integral domain.
\end{itemize}

\end{remark}

\end{tcolorbox}

~

\begin{tcolorbox}[enhanced jigsaw, colframe=quarto-callout-note-color-frame, opacityback=0, leftrule=.75mm, arc=.35mm, colback=white, rightrule=.15mm, toprule=.15mm, bottomrule=.15mm, breakable, left=2mm]

\begin{definition}[Content of a polynomial, Primitive
polynomial]\protect\hypertarget{def-default}{}\label{def-default}

\begin{itemize}
\item
  \textbf{Content of a polynomial:} The content of a polynomial
  \(a_nx^n+a_{n-1}x^{n-1} + \cdots + a_0\) is
  \(gcd(a_n,a_{n-1},\cdots, a_0)\).
\item
  \textbf{Primitive polynomial} A primitive polynnomial is an element of
  \(Z[x]\) with content \(1\).
\end{itemize}

\end{definition}

\end{tcolorbox}

\begin{tcolorbox}[enhanced jigsaw, colframe=quarto-callout-tip-color-frame, opacityback=0, leftrule=.75mm, arc=.35mm, colback=white, rightrule=.15mm, toprule=.15mm, bottomrule=.15mm, breakable, left=2mm]

\begin{theorem}[Reducibility tests for degree 2 and
3]\protect\hypertarget{thm-irreducibilityDegTwoThree}{}\label{thm-irreducibilityDegTwoThree}

Let \(F\) be a field. If \(f(x) \in F[x]\) and \(deg\;f(x)\) is 2 or 3,
then \(f(x)\) is irreducible over \(F\) iff it has a zero in \(F\).

\end{theorem}

\end{tcolorbox}

~

\begin{tcolorbox}[enhanced jigsaw, colframe=quarto-callout-tip-color-frame, opacityback=0, leftrule=.75mm, arc=.35mm, colback=white, rightrule=.15mm, toprule=.15mm, bottomrule=.15mm, breakable, left=2mm]

\begin{theorem}[Eisenstein's Criterion
1850]\protect\hypertarget{thm-eisensteinsCriterion}{}\label{thm-eisensteinsCriterion}

\hfill\break

Let \[f(x) = a_nx^n+a_{n-1}x^{n-1} + \cdots + a_0 ∈ Z[x]\]. If there is
a prime \(p\) such that \(p∤a_n, p|a_i,\cdots,p| a_0\) and
\(p^2\nmid a_0\), then \(f(x)\) is irreducible over \(Q\).

\end{theorem}

\end{tcolorbox}

\begin{tcolorbox}[enhanced jigsaw, colframe=quarto-callout-color-frame, opacityback=0, leftrule=.75mm, arc=.35mm, colback=white, rightrule=.15mm, toprule=.15mm, bottomrule=.15mm, breakable, left=2mm]

\begin{corollary}[Irreducibility of pth Cyclotomic
Polynomial]\protect\hypertarget{cor-IrreducibiltyCyclotomic}{}\label{cor-IrreducibiltyCyclotomic}

~

For any prime \(p\), the pth cyclotomic polynomial
\[\Phi_{p}(x) = \frac{x^p-1}{p-1} = x^p + x^{p-1} + \cdots + 1\] ir
irreducible over \(Q\).

\end{corollary}

\end{tcolorbox}

~

\begin{tcolorbox}[enhanced jigsaw, colframe=quarto-callout-tip-color-frame, opacityback=0, leftrule=.75mm, arc=.35mm, colback=white, rightrule=.15mm, toprule=.15mm, bottomrule=.15mm, breakable, left=2mm]

\begin{theorem}[\(⟨p(x)\rangle\) maximal iff \(p(x)\) is
irreducible]\protect\hypertarget{thm-maximaliffIrreducible}{}\label{thm-maximaliffIrreducible}

~

Let \(F\) be a field and let \(p(x) \in F[x]\). Then p(x) is irreducible
over \(F\) iff \(⟨p(x)\rangle\) is maximal over \(F\).

\end{theorem}

\end{tcolorbox}

\begin{tcolorbox}[enhanced jigsaw, colframe=quarto-callout-color-frame, opacityback=0, leftrule=.75mm, arc=.35mm, colback=white, rightrule=.15mm, toprule=.15mm, bottomrule=.15mm, breakable, left=2mm]

\begin{corollary}[F{[}x{]}/\textless p(x)\textgreater{} is a
Field]\protect\hypertarget{cor-FquotientpxField}{}\label{cor-FquotientpxField}

~

If \(p(x)\) is an irreducible polynomial over \(F\) then
\(F[x]/\langle p(x)\rangle\) is a field.

\end{corollary}

\end{tcolorbox}

\begin{tcolorbox}[enhanced jigsaw, colframe=quarto-callout-color-frame, opacityback=0, leftrule=.75mm, arc=.35mm, colback=white, rightrule=.15mm, toprule=.15mm, bottomrule=.15mm, breakable, left=2mm]

\begin{corollary}[p(x)\textbar a(x)b(x) implies p(x)\textbar a(x) or
p(x)\textbar b(x)]\protect\hypertarget{cor-DividesEither}{}\label{cor-DividesEither}

~

\(F\) field, \(p(x),a(x),b(x) \in F[x]\) such that
\(p(x)|a(x)b(x) ⟹ p(x)|a(x)\) or \(p(x)|b(x)\)

\end{corollary}

\end{tcolorbox}

~

\begin{tcolorbox}[enhanced jigsaw, colframe=quarto-callout-tip-color-frame, opacityback=0, leftrule=.75mm, arc=.35mm, colback=white, rightrule=.15mm, toprule=.15mm, bottomrule=.15mm, breakable, left=2mm]

\begin{theorem}[Unique Factorization in
Z{[}x{]}]\protect\hypertarget{thm-uniqueFactorizationZx}{}\label{thm-uniqueFactorizationZx}

~

Every non-zero non-unit polynomial in \(Z[x]\) can be expressed as a
product of irreducibles uniquely upto their order and the units.

\end{theorem}

\end{tcolorbox}

~

\subsection{Divisibility in Integral
Domains}\label{divisibility-in-integral-domains}

\begin{tcolorbox}[enhanced jigsaw, colframe=quarto-callout-note-color-frame, opacityback=0, leftrule=.75mm, arc=.35mm, colback=white, rightrule=.15mm, toprule=.15mm, bottomrule=.15mm, breakable, left=2mm]

\begin{definition}[Associates, Irreducibe,
Prime]\protect\hypertarget{def-AssociatesIrreduciblePrime}{}\label{def-AssociatesIrreduciblePrime}

~

\begin{itemize}
\item
  Associates: ~ Elements \(a\) and \(b\) in integral domain \(D\) are
  called associates if we can write \(a=bu\) for some unit \(u\) in
  \(D\).
\item
  Irreducible: ~ A non-zero element \(a\) of \(D\) is said to be
  irreducible over \(D\) if \(b,c\in D\) where \(a=bc\) implies that
  either \(b\) or \(c\) is a unit.
\item
  Prime: ~ A non-zero element \(p\) is called prime if \(a\) is not a
  unit and \(p|ab \implies p|a\) or \(p|b\).
\end{itemize}

\end{definition}

\end{tcolorbox}

TODO: Norm and its properties ~

\begin{tcolorbox}[enhanced jigsaw, colframe=quarto-callout-tip-color-frame, opacityback=0, leftrule=.75mm, arc=.35mm, colback=white, rightrule=.15mm, toprule=.15mm, bottomrule=.15mm, breakable, left=2mm]

\begin{theorem}[Prime implies
Irreducible]\protect\hypertarget{thm-PrimeImpliesIrreducible}{}\label{thm-PrimeImpliesIrreducible}

~

In an integral domain, every prime is an irreducible.

\end{theorem}

\end{tcolorbox}

\begin{tcolorbox}[enhanced jigsaw, colframe=quarto-callout-tip-color-frame, opacityback=0, leftrule=.75mm, arc=.35mm, colback=white, rightrule=.15mm, toprule=.15mm, bottomrule=.15mm, breakable, left=2mm]

\begin{theorem}[PID implies Irreducible equals
Prime]\protect\hypertarget{thm-IrreducibleEqualPrimeInPID}{}\label{thm-IrreducibleEqualPrimeInPID}

~

In a principal ideal domain, an element is irreducible if and only if it
is a prime.

\end{theorem}

\end{tcolorbox}

~

\begin{tcolorbox}[enhanced jigsaw, colframe=quarto-callout-note-color-frame, opacityback=0, leftrule=.75mm, arc=.35mm, colback=white, rightrule=.15mm, toprule=.15mm, bottomrule=.15mm, breakable, left=2mm]

\begin{definition}[Unique Factorization
Domain]\protect\hypertarget{def-UFD}{}\label{def-UFD}

~

An integral domain D is a unique factorization domain if

\begin{itemize}
\tightlist
\item
  every non zero element of D that is not a unit can be written as a
  product of irreducibles of D; and
\item
  the factorization into irreducibles is unique up to associates and the
  order in which the factor appear.
\end{itemize}

\end{definition}

\end{tcolorbox}

~

\begin{lemma}[Ascending Chain condition for
PID]\protect\hypertarget{lem-default}{}\label{lem-default}

~

In a principal ideal domain, any strictly increasing chain of ideals
\(I_1 \subset I_2 \subset \cdots\) is finite in length.

\end{lemma}

\begin{tcolorbox}[enhanced jigsaw, colframe=quarto-callout-tip-color-frame, opacityback=0, leftrule=.75mm, arc=.35mm, colback=white, rightrule=.15mm, toprule=.15mm, bottomrule=.15mm, breakable, left=2mm]

\begin{theorem}[PID implies
UFD]\protect\hypertarget{thm-PIDimpliesUFD}{}\label{thm-PIDimpliesUFD}

~

Every principal ideal domain is a unique factorization domain.

\end{theorem}

\end{tcolorbox}

\begin{tcolorbox}[enhanced jigsaw, colframe=quarto-callout-color-frame, opacityback=0, leftrule=.75mm, arc=.35mm, colback=white, rightrule=.15mm, toprule=.15mm, bottomrule=.15mm, breakable, left=2mm]

\begin{corollary}[F{[}x{]} is a
UFD]\protect\hypertarget{cor-FisUFD}{}\label{cor-FisUFD}

~

Let \(F\) be a field. Then \(F[x]\) is a unique factorization domain.

\end{corollary}

\end{tcolorbox}

~

\begin{tcolorbox}[enhanced jigsaw, colframe=quarto-callout-note-color-frame, opacityback=0, leftrule=.75mm, arc=.35mm, colback=white, rightrule=.15mm, toprule=.15mm, bottomrule=.15mm, breakable, left=2mm]

\begin{definition}[Euclidean
Domain]\protect\hypertarget{def-ED}{}\label{def-ED}

~

An integral domain \(D\) is called an Euclidean domain if there is a
function (called the measure) from the nonzero elements of \(D\) to the
nonnegative integers such that:

\begin{itemize}
\tightlist
\item
  \(d(a) \le d(ab)\) for all nonzero \(a,b\in D\); and
\item
  If \(a,b\in D, b≠ 0\), then there exist elements \(q\) and \(r\) in
  \(D\) such that \[a = bq+r\] where \(r=0\) or \(d(r) < d(b)\)
\end{itemize}

\end{definition}

\end{tcolorbox}

\begin{tcolorbox}[enhanced jigsaw, colframe=quarto-callout-tip-color-frame, opacityback=0, leftrule=.75mm, arc=.35mm, colback=white, rightrule=.15mm, toprule=.15mm, bottomrule=.15mm, breakable, left=2mm]

\begin{theorem}[ED implies
PID]\protect\hypertarget{thm-EDimpliesPID}{}\label{thm-EDimpliesPID}

~

Every Euclidean domain is a principal ideal domain.

\end{theorem}

\end{tcolorbox}

\begin{tcolorbox}[enhanced jigsaw, colframe=quarto-callout-tip-color-frame, opacityback=0, leftrule=.75mm, arc=.35mm, colback=white, rightrule=.15mm, toprule=.15mm, bottomrule=.15mm, breakable, left=2mm]

\begin{theorem}[D a UFD implies D{[}x{]} a
UFD]\protect\hypertarget{thm-default}{}\label{thm-default}

~

If \(D\) is a unique factorization domain, then \(D[x]\) is a unique
factorization domain.

\end{theorem}

\end{tcolorbox}




\end{document}
